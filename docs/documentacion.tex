Documentación de la Calculadora Aritmética Modular
1. Descripción general
Este proyecto implementa una Calculadora Aritmética Modular en Python. La calculadora realiza operaciones aritméticas básicas (suma, resta, multiplicación, división y potencia) en el contexto de la aritmética modular, donde todas las operaciones se realizan módulo un número primo p. El proyecto consta de una biblioteca de operaciones modulares y una interfaz gráfica de usuario (GUI) para facilitar su uso.

2. Estructura del código
El proyecto está organizado en dos módulos principales:

operaciones.py: Contiene las funciones para realizar operaciones aritméticas modulares.
gui.py: Implementa la interfaz gráfica de usuario utilizando Tkinter.
Módulos principales:
operaciones.py:
Funciones de validación y utilidad
Funciones de operaciones modulares
Función de evaluación de expresiones
gui.py:
Clase CalculadoraModular
Función crear_interfaz
3. Documentación detallada de funciones y métodos
operaciones.py
validate_inputs(a: int, b: int, p: int) -> None
Valida que las entradas sean enteros y que p sea un número primo positivo.

Parámetros:
a: Primer operando
b: Segundo operando
p: Módulo primo
Excepciones:
ValueError: Si algún argumento no es entero o si p no es primo
is_prime(n: int) -> bool
Verifica si un número es primo.

Parámetros:
n: Número a verificar
Retorna:
bool: True si n es primo, False en caso contrario
mod_add(a: int, b: int, p: int) -> int
Realiza la suma modular.

Parámetros:
a: Primer sumando
b: Segundo sumando
p: Módulo primo
Retorna:
int: Resultado de (a + b) mod p
Excepciones:
ValueError: Si las entradas no son válidas
mod_subtract(a: int, b: int, p: int) -> int
Realiza la resta modular.

Parámetros:
a: Minuendo
b: Sustraendo
p: Módulo primo
Retorna:
int: Resultado de (a - b) mod p
Excepciones:
ValueError: Si las entradas no son válidas
mod_multiply(a: int, b: int, p: int) -> int
Realiza la multiplicación modular.

Parámetros:
a: Primer factor
b: Segundo factor
p: Módulo primo
Retorna:
int: Resultado de (a * b) mod p
Excepciones:
ValueError: Si las entradas no son válidas
mod_divide(a: int, b: int, p: int) -> int
Realiza la división modular.

Parámetros:
a: Dividendo
b: Divisor
p: Módulo primo
Retorna:
int: Resultado de (a / b) mod p
Excepciones:
ValueError: Si las entradas no son válidas o si b es cero
mod_power(base: int, exponent: int, p: int) -> int
Realiza la exponenciación modular.

Parámetros:
base: Base
exponent: Exponente
p: Módulo primo
Retorna:
int: Resultado de (base^exponent) mod p
Excepciones:
ValueError: Si las entradas no son válidas o si el exponente es negativo
evaluate_expression(expression: str, p: int) -> int
Evalúa una expresión aritmética modular.

Parámetros:
expression: Expresión aritmética como string
p: Módulo primo
Retorna:
int: Resultado de la evaluación de la expresión mod p
Excepciones:
ValueError: Si la expresión no es válida o si p no es primo
gui.py
Clase CalculadoraModular
__init__(self, master: tk.Tk) -> None
Constructor de la clase CalculadoraModular.

Parámetros:
master: Ventana principal de Tkinter
calculate(self) -> None
Método para realizar el cálculo basado en la entrada del usuario.

Excepciones:
ValueError: Si la entrada no es válida
Exception: Para cualquier otro error inesperado
crear_interfaz() -> None
Función para crear y mostrar la interfaz gráfica de usuario.

4. Decisiones de diseño y patrones utilizados
Separación de responsabilidades: Las operaciones matemáticas están separadas de la interfaz de usuario.
Manejo de errores centralizado: Se utiliza una función de validación común para todas las operaciones.
Uso de logging: Para facilitar la depuración y el seguimiento de errores.
Interfaz gráfica simple: Utilizando Tkinter para una fácil interacción del usuario.
5. Requisitos del sistema y dependencias externas
Python 3.6 o superior
Tkinter (generalmente incluido en las instalaciones estándar de Python)
6. Instrucciones de instalación y configuración
Asegúrese de tener Python 3.6 o superior instalado.
Clone o descargue los archivos operaciones.py y gui.py.
No se requieren pasos adicionales de instalación.
7. Ejemplos de uso

# Ejemplo de uso de operaciones modulares
print(mod_add(5, 3, 7))  # Salida: 1
print(mod_subtract(5, 3, 7))  # Salida: 2
print(mod_multiply(5, 3, 7))  # Salida: 1
print(mod_divide(5, 3, 7))  # Salida: 4
print(mod_power(5, 3, 7))  # Salida: 6

# Para usar la interfaz gráfica
python gui.py
8. Manejo de errores y validación de entradas
Se utilizan bloques try-except para manejar excepciones específicas.
La función validate_inputs verifica que las entradas sean enteros y que p sea primo.
Se implementa logging para registrar errores y operaciones exitosas.
La GUI muestra mensajes de error al usuario mediante cuadros de diálogo.
9. Seguridad y mejores prácticas
Validación de entradas para prevenir inyección de código malicioso.
Uso controlado de eval() para la evaluación de expresiones.
Logging para auditoría y depuración.
10. Pruebas
Se incluyen ejemplos básicos de prueba en operaciones.py.
Se recomienda implementar pruebas unitarias más exhaustivas utilizando un framework como pytest.
11. Sugerencias para mejoras futuras
Implementar un parser más robusto para expresiones matemáticas complejas.
Agregar soporte para operaciones más avanzadas (e.g., logaritmos modulares, raíces cuadradas modulares).
Mejorar la interfaz gráfica con más opciones y una mejor presentación de resultados.
Implementar un historial de cálculos.
12. Glosario de términos
Aritmética modular: Sistema aritmético para clases de equivalencia de números enteros.
Módulo: Número que define el conjunto sobre el cual se realizan las operaciones modulares.
Número primo: Número natural mayor que 1 que solo es divisible por 1 y por sí mismo.
Inverso multiplicativo modular: Número que, cuando se multiplica por otro número módulo p, da como resultado 1.
Esta documentación proporciona una visión completa del proyecto de la Calculadora Aritmética Modular, incluyendo su estructura, funcionamiento, y consideraciones de diseño e implementación. Es una herramienta valiosa para entender, utilizar y potencialmente extender el proyecto en el futuro.